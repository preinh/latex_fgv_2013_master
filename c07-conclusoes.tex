%% ------------------------------------------------------------------------- %%
\chapter{Discussão}
\label{cap:conclusoes}

Apesar de possuirem implementações distintas de um mesmo fundamento matemático e estatítstico
que é a estimativa de densidade por funções de núcleo, os métodos, no geral, conseguiram
estimar das características da sismicidade a partir da sismicidade histórica catalogada.

Alguns distribuiram mais a taxa de sismicidade e a ameaça, uns destacando certas feições, outros outras.
Pode-se perceber que a caracterização da distribuição de frequência e magnitude, o uso da sua forma discreta 
ou truncada, pode ter influência no valor calculado para a ameaça e merece atenção.

Não parece haver um modelo mais correto, mesmo
porque todas as formulações são coerentes ao que se propuseram. Mesmo assim
os resultados, considerando suas diferentes proposições variam bastante.
O modelo de projeção de ocorrência de rupturas de Helmstetter talvez nem
fosse aplicável num universo de poucas amostras distribuídas em uma área tão vasta.

Essa caracterização suavizada da taxa de sismicidade, mesmo que executada de forma
distinta por critérios distintos, certamente terá relevãncia e deverá ser considerada
de alguma forma pelos sismólogos ou engenheiros modeladores de ameaça sísmica,
mesmo que apenas para cumprir certo percentual de uma árvore lógica.

Ficou claro também, mesmo não tendo sido o foco específico desse trabalho, 
não ser possível negligenciar a ocorrência de sismos profundos da zona de 
subducção para a ameaça sísmica do extremo oeste do país. Mesmo os sismos muito profundos (da ordem de centenas de km)
não provocando grandes impactos nas estruturas, sismos maiores e um pouco mais distantes podem ter razoáveis amplitudes em determinados perídos do espectro de aceleração.

O openquake como calculador de ameaças foi positivo do ponto de vista metodológico,
e o material bem documentado serviu de apoio e esclarecimento do cálculo.

O resultado discrepante entre os valores apresentados pelo Crisis-2007 e o Openquake,
precisam ser investigados com maior detalhe, mas os resultados do openquake foram compatíveis com o
modelo global, mesmo os valores calculados com o mesmo modelo de fontes sísmicas.

O HMTK se mostrou uma ferramenta essencial para a modelagem da ameaça. 
Muitas funcionalidades permitem facilmente a implementação dos principais fluxos de trabalho.

Por fim, talvez seja possível dizer que os objetivos propostos foram cumpridos.

%------------------------------------------------------
\section{Considerações Finais} 

Foi possível de certa forma aplicar as técnicas de suavização, estimadoras das taxas suavizadas de sismicidade, 
gerar uma grade regular de fontes sísmicas pontuais, e então calcular a ameaça sísmica obtendo 
valores razoáveis, sem com isso definir zonas sísmicas, é o que se conhece também como métodos de \emph{zoneless}.

As diferenças nas formas de escolher a largura de banda das funções de núcleo de cada método,
podem talvez render ao método de Hemlstetter alguma vantagem, por ser localmente adaptável,
quando define claramente feições no nordeste. as outros métodos, com escolhas mais rígidas,
também o fizeram, privilegiando outras regiões.

O ajuste do método de Woo para o Brasil forneceu enormes larguras de banda, com cerca de 1500km para
as maiores magnitudes (são poucas e o método se baseia em vizinhos mais próximos), isso dilui
a influência dos grandes sismos nas taxas de sismicidade.

No caso do Brasil, essas primeiras estimativas sugerem que estudos de maior detalhe 
sejam feitos nas regiões de maior destaque.



%------------------------------------------------------
\section{Sugestões para Pesquisas Futuras} 

Ainda há muito o que explorar.

A variação temporal e espacial da magnitude de completude no Brasil seria a principal fonte de informação 
para melhoria significativa da resposta dos métodos que dependem dessas correções para darem bons resultados.
Seu conhecimento permite também um mapeamento dos \emph{valores-b}. \citet{vorobieva_2013} aponta caminhos muito
interessantes.

Consideração de modelos que levem em consideração a distribuição da perda de tensão na forma de momento sísmico,
e mesmo a suavização da distribuição de momento acumulado espacialmente pela placa também podem ser outra alternativa.

Na modelagem da ameaça sísmica, um passo importante, seria trabalhar daqui por diante também com uma metodologia
para seleção das relações de atenuação.

Nas regiões de maior sismicidade, talvez seja possível estudar modelos onde a taxa de sismicidade varia com o tempo
como os modelos de sequência epidêmica de pós-abalos, que podem dar melhor resposta com pouco volume de dados.

Coletar incertezas e elaborar um cenário de ameaça,
detalhando a árvore lógica de possibilidades e variações
tanto nos modelos de fonte sísmica, incluíndo eventos históricos característicos, 
como nos modelos de atenuação. 

Estudos de desagregação \citep{pagani_2007} também precisarão ser feitos no futuro.

