%-----------------------------------------
% terms
%-----------------------------------------
\newglossaryentry{equake}
{
	name={terremoto},
	description={ruptura de alguma estrutura geológica},
	plural={terremotos}
}

\newglossaryentry{hypocenter}
{
	name={hipocentro},
	description={representação geométrica do ponto no espaço e no tempo, onde 
		se iniciou o \gls{rupture_process} da \gls{crust}},
	plural={hipocentros}
}

\newglossaryentry{epicenter}
{
	name={epicentro},
	description={projeção ortogonal sobre a superfície do \gls{hypocenter}},
	plural={epicentros}
}


\newglossaryentry{seismotectonic}
{
	name={sismotectônica},
	description={o estudo das relações entre os \glspl{equake} e a \gls{tectonic} recente de uma região.
				 procuram entender quais mecanismo das rupturas na geologia são responsáveis pela \gls{seismic_activity}
				 em uma certa área, analisando de forma combinada, registros recentes de tectonismo regional e considerando também
				 evidências históricas e geomorfológicas},
}


\newglossaryentry{rupture_process}
{
	name={processo de ruptura},
	description={processo que envolve o rompimento de uma região da crosta,
			o deslocamento relativo entre essas regiões, e consequantemente,
			a liberação de uma grande quantidade de energia, de forma praticamente
			instantânea, tomando-se como referência o \gls{geologic_time}},
	plural={processos de ruptura}
}


\newglossaryentry{geologic_time}
{
	name={tempo geológico},
	description={escala de tempo que vai desde a formação do universo até os tempos atuais,
				englobando a formação do planeta e as transformações ocorridas desde então},
}


\newglossaryentry{tectonic}
{
	name={tect\^onica},
	description={ramo da ci\^{e}ncia que estuda os processos respons\'aveis 
				 pela cria\c{c}\~ao e transforma\c{c}\~ao dos planetas, 
				 especialmente a Terra},
	plural={tect\^onicas}
}


\newglossaryentry{crust}
{
	name={crosta terrestre},
	description={parte superficial, rígida e mais externa do planeta Terra},
}

\newglossaryentry{mantle}
{
	name={manto terrestre},
	description={material da por{ç}{ã}o intermediária do planeta, 
		fluido em tempo geológico},
}

\newglossaryentry{core}
{
	name={n{ú}cleo terrestre},
	description={por{ç}{ã}o mais central do planeta, com predomin{â}ncia de compostos metálicos},
}

\newglossaryentry{tectonic_plate_theory}
{
	name={teoria tect{ô}nica das placas},
	description={foi uma teoria revolucionária para a \gls{tectonic},
				propondo que a \gls{crust} terrestre estivesse dividida 
				em placas {à} deriva sobre o \gls{mantle}},
}


\newglossaryentry{litho_plate}
{
	name={placa litosf{é}rica},
	plural={placas litosf{é}ricas},
	description={placa de material da \gls{lithosphere}},
}


\newglossaryentry{lithosphere}
{
	name={litosfera},
	description={região rúptil, mais externa do planeta, formada pela \gls{crust} 
		(continental e ocêanica) e parte do \gls{mantle} superior, com aproximadamente 
		60\gls*{sym:km} de profundidade},
}


\newglossaryentry{astenosphere}
{
	name={astenosfera},
	description={região dúctil entre a \gls{lithosphere} e o \gls{mantle},
				com profundidades que variam de 60 a 700\gls*{sym:km}},
}

\newglossaryentry{smoothing}
{
	name={técnicas de suavização},
	description={consiste em capturar importantes feições do conjunto de dados,
				 eliminando ruídos e outras estruturas de curto comprimento de onda
				 presentes nos dados},
}

\newglossaryentry{kernel_function}
{
	name={função de kernel},
	description={funções n-dimensionais, cuja integral em todo o domínio resulta em 1,
				 podendo ser usadas como estimativas para 
				 funções de densidade de probabilidade},
	plural={funções de kernel},
	see={\gls{pdf}}
}

\newglossaryentry{seismic_rate}
{
	name={taxa de sismicidade},
	description={taxa com que terremotos são produzidos por determinada \gls{seismic_source}},
	plural={taxas de sismicidade},
	see={seismic_source, equake}	
}

\newglossaryentry{seismic_activity}
{
	name={atividade sísmica},
	description={frequ{ê}cia de ocorr{ê}ncia de \glspl{equake}},
	see={\gls{seismic_source}}	  
}

\newglossaryentry{pdf}
{
	name={pdf},
	description={função de densidade de probabilidade},
}

\newglossaryentry{poisson_process}
{
	name={processo de Poisson},
	description={uma sequencia de experimentos de Bernoulli com taxa \gls{sym:lambda}},
	see={\glsdesc{sym:lambda}}
}

\newglossaryentry{seismic_source}
{
	name={fonte sísmica},
	description={estrutura geológica capaz de produzir tremores de terra},
	plural={fontes sísmicas},
	see={equake}	
}

\newglossaryentry{point_source}
{
	name={fonte sísmica pontual},
	description={representação geométrica por um ponto, de uma fonte sísmica},
	plural={fontes sísmicas pontuais},
	see={seismic_source}	
}


\newglossaryentry{area_source}
{
	name={fonte sísmica poligonal},
	description={representação geométrica por um poligono em súperfície, 
				 de uma fonte sísmica},
	plural={fontes sísmicas poligonais},
	see={seismic_source}	
}

\newglossaryentry{titulo_da_dissertacao}
{
	name={titulo_da_dissertacao},
	description={Técnicas de suavização aplicadas
					à caracterização de fontes sísmicas e 
					à análise probabilistica de ameaça sísmica},
}

