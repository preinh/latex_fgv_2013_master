%% ------------------------------------------------------------------------- %%
\chapter{Contexto Teórico}
\label{cap:contexto}



Texto texto texto texto texto texto texto texto texto texto texto texto texto
texto texto texto texto texto texto texto texto texto texto texto texto texto
texto texto texto texto texto texto texto texto texto texto texto texto texto


%% ------------------------------------------------------------------------- %%
\section{Tectônica}\index{área do
trabalho!fundamentos}
\label{sec:fundamentos}

%% ------------------------------------------------------------------------- %%
\subsection{Teoria de Placas}\index{área do
trabalho!fundamentos}
\label{sec:fundamentos}

Texto texto texto texto texto texto texto texto texto texto texto texto texto
texto texto texto texto texto texto texto texto texto texto texto texto texto
texto texto texto texto texto texto texto texto texto texto texto texto texto
texto texto texto texto texto texto texto texto texto texto texto texto texto
texto texto texto.


%% ------------------------------------------------------------------------- %%
\subsubsection{Bordas}\index{área do
trabalho!fundamentos}
\label{sec:fundamentos}

Texto texto texto texto texto texto texto texto texto texto texto texto texto
texto texto texto texto texto texto texto texto texto texto texto texto texto
texto texto texto texto texto texto texto texto texto texto texto texto texto
texto texto texto texto texto texto texto texto texto texto texto texto texto
texto texto texto.

%% ------------------------------------------------------------------------- %%
\subsubsection{Interior}\index{área do
trabalho!fundamentos}
\label{sec:fundamentos}

Texto texto texto texto texto texto texto texto texto texto texto texto texto
texto texto texto texto texto texto texto texto texto texto texto texto texto
texto texto texto texto texto texto texto texto texto texto texto texto texto
texto texto texto texto texto texto texto texto texto texto texto texto texto
texto texto texto.

%% ------------------------------------------------------------------------- %%
\subsection{Sismotectônica}\index{área do
trabalho!fundamentos}
\label{sec:fundamentos}

Texto texto texto texto texto texto texto texto texto texto texto texto texto
texto texto texto texto texto texto texto texto texto texto texto texto texto
texto texto texto texto texto texto texto texto texto texto texto texto texto
texto texto texto texto texto texto texto texto texto texto texto texto texto
texto texto texto.



\section{Sismicidade}\index{área do
trabalho!fundamentos}
\label{sec:risco_sismico}

A sismicidade é o estudo da frequência de ocorrência de tremores de terra, no
tempo e no espaço.

Tremores de terra, abalos, terremotos, sismos são a ocorrência de
fenômenos geológicos de ruptura 'instanânea', com um certo tamanho, na
crosta terrestre.


\subsection{Ocorrência}\index{área do
trabalho!fundamentos}
\label{sec:risco_sismico}
 
A ocorrência dos tremores se dá num tempo \ac{$t$} e num lugar \ac{$r$} da
crosta que pode ser considerado simplificadamente ora como
bi-dimensional: o eapaço dos \emph{epicentros} $ E = {(x,y)}$, ora como
tri-dimensional: o espaço dos \emph{hipocentros} $ H = {(x,y,z)}$.

Na prática o que ocorre realmente é o deslocamento relativo (\emph{$rake$})
entre estruturas geologias em uma certa área \emph{$A_{rupture}$}, o falhamento geológico.


\subsubsection{Processo de Poisson}\index{área do
trabalho!fundamentos}
\label{sec:risco_sismico}

Definição do processo\ldots

Críticas\ldots


\subsubsection{Independência entre eventos}\index{área do
trabalho!fundamentos}
\label{sec:risco_sismico}

Pressupostos\ldots

Críticas\ldots


\subsubsection{Omori-Utsu}\index{área do
trabalho!fundamentos}
\label{sec:risco_sismico}

Definição do processo\ldots

Críticas\ldots


\subsection{Falhamento Geológico}\index{área do
trabalho!fundamentos}
\label{sec:risco_sismico}

De forma simplificada, os falhamentos geológicos podem ser classificados em três
grupos segundo a teoria da tecônica de placas, que versa sobre a interação
entre crosta e o manto terrestre e seus reflexos na geologia crustal:

\subsubsection{Falhamento Normal}\index{área do
trabalho!fundamentos}
\label{sec:risco_sismico}

Ou seja que ocorrem em regime de compressão


\subsubsection{Falhamento Reverso}\index{área do
trabalho!fundamentos}
\label{sec:risco_sismico}

Ou seja, por distenção, fenômeno mais raro como a abertura de oceanos 


\subsubsection{Falhamento Transcorrente/Transverso}\index{área do
trabalho!fundamentos}
\label{sec:risco_sismico}

O que ocorre de maneira oblíqua.



 
\subsection{Predição da Ocorrência de Rupturas}\index{área do
trabalho!fundamentos}
\label{sec:fundamentos}

Texto texto texto texto texto texto texto texto texto texto texto texto texto
texto texto texto texto texto texto texto texto texto texto texto texto texto
texto texto texto texto texto texto texto texto texto texto texto texto texto


\subsubsection{Curto-prazo}\index{área do trabalho!fundamentos}
\label{sec:fundamentos}


Texto texto texto texto texto texto texto texto texto texto texto texto texto
texto texto texto texto texto texto texto texto texto texto texto texto texto
texto texto texto texto texto texto texto texto texto texto texto texto texto


\subsubsection{Longo-prazo}\index{área do trabalho!fundamentos}
\label{sec:fundamentos}

Texto texto texto texto texto texto texto texto texto texto texto texto texto
texto texto texto texto texto texto texto texto texto texto texto texto texto
texto texto texto texto texto texto texto texto texto texto texto texto texto
texto texto texto texto texto texto texto texto texto texto texto texto texto
texto texto texto texto texto texto.


 
\subsection{Magnitude - Tamanho da Ruptura}\index{área do
trabalho!fundamentos}
\label{sec:risco_sismico}

A magnitude de um tremor de terra, é um valor medido numa escala que busca
refletir a energia liberada pelo tremor de terra como sendo proporcional à área
e ao deslocamento da ruptura geológica que o originou.

O desenvolvimento das escalas de magnitude para medir o tamanho dos tremores,
iniciou-se experimentalmente com Richter e segue até sua corrente definição



\subsubsection{Magnitude Richter }\index{área do
trabalho!fundamentos}
\label{sec:risco_sismico}

O desenvolvimento das escalas de magnitude para medir o tamanho dos tremores,
iniciou-se experimentalmente com Richter e segue até sua corrente definição


\subsubsection{Magnitude de Momento Sísmico }\index{área do
trabalho!fundamentos}
\label{sec:risco_sismico}

O desenvolvimento das escalas de magnitude para medir o tamanho dos tremores,
iniciou-se experimentalmente com Richter e segue até sua corrente definição





\subsubsection{Intensidade Macrossísmica}
\index{área do trabalho!fundamentos}
\label{sec:risco_sismico}


A intensidade macrossísmica é uma escala para medir, não a energia proporcional
à ruptura que originou o tremor de terra, mas para retratar a percepção do
movimento do chão onde quer tenha produzido seus efeitos perceptíveis ao ser
humano.

Uma das mais difundidas é a escala de Mercalli:


TABELA


Existem estudos que propõem a inferência sobre o tamanho da ruptura, e sua
magnitude, a partir de observações macrossísmicas, ou relatos georreferenciados.






%% ------------------------------------------------------------------------- %%
\subsection{Representação}\index{área do trabalho!fundamentos}
\label{sec:fundamentos}


%% ------------------------------------------------------------------------- %%
\subsubsection{Epicentros}\index{área do trabalho!fundamentos}
\label{sec:fundamentos}

Epicentros $ E : {(x,y,t,m)}$

Com incertezas

$ E : {(x,\sigma_x,y,\sigma_y,t,\sigma_t,m,\sigma_m)}$


%% ------------------------------------------------------------------------- %%
\subsubsection{Hipocentros}\index{área do trabalho!fundamentos}
\label{sec:fundamentos}

hipocentros $ H : {(x,y,z,t,m)}$

Com incertezas

hipocentros $ H : {(x,\sigma_x,y,\sigma_y,z,\sigma_z,t,\sigma_t,m,\sigma_m)}$


%% ------------------------------------------------------------------------- %%
\subsubsection{Falhamentos}\index{área do trabalho!fundamentos}
\label{sec:fundamentos}

Falhamentos ocorrem como deslocamentos em planos 

$ F : {(d,s,r)}$

Dip, Strike, Rake



%% ------------------------------------------------------------------------- %%
\subsubsection{Rupturas}\index{área do trabalho!fundamentos}
\label{sec:fundamentos}

Falhamentos ocorrem como deslocamentos como tensores 

$ F : {(d,s,r)}$




\subsection{Distribuição de Frequência e Magnitudes}\index{área do
trabalho!fundamentos}
\label{sec:risco_sismico}

À uma primeira abordagem estatística, seria conveniente analisar a frequência de
ocorrencia dos sismos.

\subsubsection{Gutemberg-Richter MFD}\index{área do
trabalho!fundamentos}
\label{sec:risco_sismico}

Gutemberg e Richter, por terem desenvolvido à escala de tamanho dos
tremores, foram os primeiros a analisar sua frequência de ocorrencia com estes
mesmos tamanhos.

A observação de que a forma da sua distribuição se caracterizava por uma 
XXXXX, é utilizada até hoje para a estimativa da ocorrencia de tremores futuros.



\subsubsection{Truncated MFD}\index{área do
trabalho!fundamentos}
\label{sec:risco_sismico}

Como variações ou critérios de contorno, algumas distribuições foram surgindo,
como por exemplo a distribuição truncada de magnitude e frequência (TMFD):


FORMULAS


GRAFICO



\subsubsection{Tapered MFD}\index{área do
trabalho!fundamentos}
\label{sec:risco_sismico}


Existe também distribuições que terminam suavemente em magnitudes menos
elevadas, sugerindo a raridade dos fenomenos geológicos ou o tempo escasso de
observação.


\subsubsection{Exponential cutoff MFD}\index{área do
trabalho!fundamentos}
\label{sec:risco_sismico}

Entretanto, fenômenos de maiores magnitudes podem ter sido noticiados em
maior quantidade sugerindo distribuições um pouco mais alongadas:

GRAFICO

FORMULA



\section{Taxa de Sismicidade}\index{área do
trabalho!fundamentos}
\label{sec:risco_sismico}


Texto texto texto texto texto texto texto texto texto texto texto texto texto
texto texto texto texto texto texto texto texto texto texto texto texto texto
texto texto texto texto texto texto texto texto texto texto texto texto texto


\subsubsection{Magnitude de Completude}\index{área do
trabalho!fundamentos}
\label{sec:risco_sismico}

A magnitude de completude\ldots 

texto texto texto texto texto texto texto texto texto texto texto texto texto


GRAFICO

FORMULA



\subsubsection{Valor-b}\index{área do
trabalho!fundamentos}
\label{sec:risco_sismico}


Texto texto texto texto texto texto texto texto texto texto texto texto texto
texto texto texto texto texto texto texto texto texto texto texto texto texto
texto texto texto texto texto texto texto texto texto texto texto texto texto


\subsubsection{Valor-a}\index{área do
trabalho!fundamentos}
\label{sec:risco_sismico}



Texto texto texto texto texto texto texto texto texto texto texto texto texto
texto texto texto texto texto texto texto texto texto texto texto texto texto
texto texto texto texto texto texto texto texto texto texto texto texto texto




\section{Risco Sísmico}\index{área do trabalho!fundamentos}
\label{sec:risco_sismico}

-> risk = ( hazard, 
			exposition, 
			vulnerability)


Texto texto texto texto texto texto texto texto texto texto texto texto texto
texto texto texto texto texto texto texto texto texto texto texto texto texto
texto texto texto texto texto texto texto texto texto texto texto texto texto



\section{Ameaça Sísmica}\index{área do trabalho!fundamentos}
\label{sec:ameaca_sismica}

-> hazard = seismic sources, 
			rupture occurence, 
			ground motion prediction 



Texto texto texto texto texto texto texto texto texto texto texto texto texto
texto texto texto texto texto texto texto texto texto texto texto texto texto
texto texto texto texto texto texto texto texto texto texto texto texto texto



\section{Análise Probabilistica de Ameaça Sísmica}\index{área do
trabalho!fundamentos}
\label{sec:psha}


cornell, mcguire ?!?


Texto texto texto texto texto texto texto texto texto texto texto texto texto
texto texto texto texto texto texto texto texto texto texto texto texto texto
texto texto texto texto texto texto texto texto texto texto texto texto texto



%% ------------------------------------------------------------------------- %%
\subsection{Caracterização de Fontes Sísmicas}
\index{ácido!nucléico}\index{nucleotídeos}
\label{sec:fontes}

cornell, mcguire ?!?


Texto texto texto texto texto texto texto texto texto texto texto texto texto
texto texto texto texto texto texto texto texto texto texto texto texto texto
texto texto texto texto texto texto texto texto texto texto texto texto texto



\subsubsection{Tipologia e Representação Geométrica}
\index{ácido!nucléico}\index{nucleotídeos}
\label{sec:fontes_tipologia}


\subsubsubsection{Falha Complexa}
\index{ácido!nucléico}\index{nucleotídeos}
\label{sec:fonte_falha_complexa}


Texto texto texto texto texto texto texto texto texto texto texto texto texto
texto texto texto texto texto texto texto texto texto texto texto texto texto
texto texto texto texto texto texto texto texto texto texto texto texto texto



\subsubsubsection{Falha Simples}
\index{ácido!nucléico}\index{nucleotídeos}
\label{sec:fonte_falha_complexa}



Texto texto texto texto texto texto texto texto texto texto texto texto texto
texto texto texto texto texto texto texto texto texto texto texto texto texto
texto texto texto texto texto texto texto texto texto texto texto texto texto



\subsubsubsection{Área}
\index{ácido!nucléico}\index{nucleotídeos}
\label{sec:fonte_falha_complexa}


Texto texto texto texto texto texto texto texto texto texto texto texto texto
texto texto texto texto texto texto texto texto texto texto texto texto texto
texto texto texto texto texto texto texto texto texto texto texto texto texto



\subsubsubsection{Pontos}
\index{ácido!nucléico}\index{nucleotídeos}
\label{sec:fonte_falha_complexa}


Texto texto texto texto texto texto texto texto texto texto texto texto texto
texto texto texto texto texto texto texto texto texto texto texto texto texto
texto texto texto texto texto texto texto texto texto texto texto texto texto



\subsubsection{Caracterização}
\index{ácido!nucléico}\index{nucleotídeos}
\label{sec:fontes}



Texto texto texto texto texto texto texto texto texto texto texto texto texto
texto texto texto texto texto texto texto texto texto texto texto texto texto
texto texto texto texto texto texto texto texto texto texto texto texto texto



\subsubsubsection{Ocorrência de Sismicidade}
\index{ácido!nucléico}\index{nucleotídeos}
\label{sec:fontes}


Texto texto texto texto texto texto texto texto texto texto texto texto texto
texto texto texto texto texto texto texto texto texto texto texto texto texto
texto texto texto texto texto texto texto texto texto texto texto texto texto


\subsubsubsection{Ocorrência de Sismicidade}
\index{ácido!nucléico}\index{nucleotídeos}
\label{sec:fontes}




Texto texto texto texto texto texto texto texto texto texto texto texto texto
texto texto texto texto texto texto texto texto texto texto texto texto texto
texto texto texto texto texto texto texto texto texto texto texto texto texto



\subsubsection{Caracterização de Fontes Sísmicas}
\index{ácido!nucléico}\index{nucleotídeos}
\label{sec:fontes}



Texto texto texto texto texto texto texto texto texto texto texto texto texto
texto texto texto texto texto texto texto texto texto texto texto texto texto
texto texto texto texto texto texto texto texto texto texto texto texto texto





%% ------------------------------------------------------------------------- %%
\subsection{Predição do Movimento do Chão} 
\index{ácido!nucléico}\index{nucleotídeos}
\label{sec:gmpe}


Texto texto texto texto texto texto texto texto texto texto texto texto texto
texto texto texto texto texto texto texto texto texto texto texto texto texto
texto texto texto texto texto texto texto texto texto texto texto texto texto

