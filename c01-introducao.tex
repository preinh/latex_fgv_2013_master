%% ------------------------------------------------------------------------- %%
\chapter{Introdução}
\label{cap:introducao}

Um elemento primordial na análise
de \emph{risco} sísmico é a análise da \emph{ameaça} sísmica,
onde a identificação e caracterização das fontes sismogênicas (causadoras de
movimento do chão, fundamentalmente tremores de terra) é a primeira das etapas. 

Considera-se nessa fase, principalmente as falhas
geológicas, o acúmulo de tensão medido através o movimento relativo da crosta
terrestre, a neotecnônica da crosta, o possível acoplamento entre placas, os tremores
(falhamentos) já registrados anteriormente, enfim, todo conhecimento geológico
disponível, para caracterizar (a) a geometria espacial da feição geológica e provável fonte
sísmica e (b) o número de ocorrência - taxa - dos tremores conforme a
proporção em energia liberada - magnitude.

No Brasil, onde a ocorrência de tremores não é desprezível mas menor que a de
outras partes do planeta, o processo de identificação das fontes sísmicas é
executado geralmente através da opinão de especialistas que fazem o zoneamento
sísmico segundo as informações técnicas e a experiência que dispõem.

Para cada uma dessas zonas sísmicas, que serão consideradas como tendo atividade
sísmica uniforme, é calculada a distribuição da ocorrência de tremores em função
da magnitude de cada tremor (e normalizada pela área?!).

Existem entretanto diversas propostas metológicas envolvendo a suavização através de 
estimativas da taxa de sismicidade por funções de núcleo, entre outras, a de
\citet{frankel_mapping_1995}, a de \citet{woo_1996} e a de
\citet{helmstetter_2012} abordadas, aqui, com maior detalhe.

O que todas elas possuem em comum é o objetivo de caracterizar a taxa de
sismicidade (ocorrência de tremores) em uma malha sobre a região de interesse
através da soma da contribuição de funções de núcleo - gaussianas, leis de
potência, entre outros - em cada nó dessa malha. O pressuposto central dessa
idéia é que os sismos (principalmente os grandes, com menor evidência, pois
aconteram menos fenômenos observáveis desse tipo) tendem a ocorrer no entorno de
onde já ocorreram antes outros tremores (menores e mais frequentes).

Fundamentalmente, o que os diferencia é a forma de escolher a largura dessas
funções de núcleo associadas à cada tremor do catálogo.

O que se pretende aqui é observar um pouco mais detalhademente o comportamento
desses diferentes métodos num ambiente com baixa e esparsa sismicidade.

Perifericamente, aproveitou-se a oportunidade para avaliar um recente
conjunto de programas de computador disponibilizado com código livre voltado à
esse segmento.

\begin{small}
\begin{verbatim}
Modos de citação:
indesejável: (Andrew e Foster, 1983) introduziram o algoritmo ótimo.
certo : Andrew e Foster introduziram o algoritmo ótimo (Andrew e Foster, 1983).
\end{verbatim}
\end{small}



%% ------------------------------------------------------------------------- %%
\section{Considerações Preliminares}
\label{sec:consideracoes_preliminares}

Considerações preliminares\index{genoma!projetos}.
% index permite acrescentar um item no indice remissivo
Texto texto texto texto texto texto texto texto texto texto texto texto texto
texto texto texto texto texto texto texto texto texto texto texto texto texto
texto texto texto texto texto texto texto.
 

%% ------------------------------------------------------------------------- %%
\section{Objetivos}
\label{sec:objetivo}

O principal objetivo desenvolvido ao longo desse trabalho é avaliar a
aplicabilidade das técnicas suavização (das antigas às mais recentes) para a
caracterização da ocorrência de sismos no Brasil.

Secundariamente, aproveita-se a oportunidade para testar o uso de um conjunto
recente de programas de computador disponível livremente, o \emph{OpenQuake}. 

%% ------------------------------------------------------------------------- %%
\section{Contribuições}
\label{sec:contribucoes}

As principais contribuições deste trabalho são:

\begin{itemize}
  \item Dispôr sobre métodos alternativos para a caracterização de fontes
  sismogênicas, a primeira das etapas da análise probabilística de risco
  sísmico.

  \item Compreender parte o \emph{OpenQuake}, um conjunto de programas de
  computador desenvolvido recentemente e oferecido com código livre pela Fundação
  GEM\footnote{Global Earthquake Modeling, Pavia, Italia.}. 
  
  \item Implementar parte dos métodos utilizados no contexto do
  \emph{OpenQuake}, ampliando os recursos oferecidos e deixando-os disponíveis
  para uso futuro de forma integrada.

\end{itemize}

%% ------------------------------------------------------------------------- %%
\section{Organização do Trabalho}
\label{sec:organizacao_trabalho}

No Capítulo~\ref{cap:conceitos}, apresentamos os conceitos ... Finalmente, no
Capítulo~\ref{cap:conclusoes} discutimos algumas conclusões obtidas neste
trabalho. Analisamos as vantagens e desvantagens do método proposto ... 

As sequências testadas no trabalho estão disponíveis no Apêndice \ref{ape:sequencias}.
