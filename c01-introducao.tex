%% ------------------------------------------------------------------------- %%
\chapter{Introdução}
\label{cap:introducao}

Um elemento primordial na análise
de \emph{risco} sísmico é a análise da \emph{ameaça} sísmica,
onde a identificação e caracterização das fontes sismogênicas (causadoras de
movimento do chão, fundamentalmente tremores de terra) é a primeira das etapas. 

Considera-se nessa fase, principalmente as falhas
geológicas, o acúmulo de tensão medido através o movimento relativo da crosta
terrestre, a neotecnônica da crosta, o possível acoplamento entre placas, os tremores
(rupturas e falhamentos) já registrados anteriormente, enfim, todo conhecimento geológico
disponível, para caracterizar (a) a geometria espacial da feição geológica e provável fonte
sísmica e (b) o número de ocorrência - taxa - dos tremores conforme a
proporção de energia liberada por cada um - magnitude.

No Brasil, onde a ocorrência de tremores não é desprezível mas menor do que a de
outras partes do planeta, o processo de identificação das fontes sísmicas é
executado geralmente através da opinão de especialistas que fazem o zoneamento
sísmico segundo as informações técnicas e a experiência que dispõem.

Para cada uma dessas zonas sísmicas, que serão consideradas como tendo atividade
sísmica uniforme, é determinada a distribuição da ocorrência de tremores em função
da magnitude de cada tremor.

Existem também outras propostas metológicas envolvendo técnicas de suavização
que permitem estimativas da taxa de sismicidade, por exemplo por funções de núcleo.
As propostas de \citet{frankel_1995}, a de \citet{woo_1996} e a de
\citet{helmstetter_2012} serão discutidas com maior detalhe.

O que todas elas possuem em comum é o objetivo de caracterizar a taxa de
sismicidade (ocorrência de tremores) em uma malha sobre a região de interesse
através da soma da contribuição de funções de núcleo (gaussianas, leis de
potência, etc) em cada nó dessa malha.

O pressuposto central dessa
idéia é que os grandes sismos (com menos evidências, pois
aconteram poucos fenômenos observáveis desse tipo) 
tendem a ocorrer no entorno de
onde já ocorreram antes outros tremores (menores e mais frequentes).

Fundamentalmente o que os diferencia é a forma de escolher a largura para dessas
funções de núcleo associadas à cada tremor do catálogo.

O que se pretende é observar um pouco mais detalhademente o comportamento
desses diferentes métodos num ambiente com baixa e esparsa sismicidade.

Perifericamente, aproveitou-se a oportunidade para avaliar um recente
conjunto de programas de computador, com código livre, para esse tipo de análise.
 

%% ------------------------------------------------------------------------- %%
\section{Objetivos}
\label{sec:objetivo}

O principal objetivo pretendido é a avaliação da
aplicabilidade de algumas técnicas de suavização para a
caracterização da ocorrência de sismos no Brasil.

Secundariamente, aproveita-se a oportunidade para testar o uso de um conjunto
recente de programas de computador disponível livremente pela e para a comunidade
científica, o Openquake\footnote{\url{http://www.globalquakemodel.org/openquake}}. 

Perifericamente, talvez seja possível fazer alguma contribuição ao Openquake
ou a alguma outra biblioteca associada.

%% ------------------------------------------------------------------------- %%
\section{Contribuições}
\label{sec:contribucoes}

As principais contribuições deste trabalho são:

\begin{itemize}
  \item Discorrer sobre métodos alternativos ao zoneamento para a caracterização de fontes
  sismogênicas, a primeira das etapas da análise probabilística de ameaça 
  sísmica.

  \item Compreender e assimilar como usar Openquake, um conjunto de programas de
  computador desenvolvido recentemente e oferecido com código livre. 
  
  \item Implementar parte dos métodos utilizados no contexto do
  Openquake, ampliando os recursos oferecidos e deixando-os disponíveis
  para uso futuro de forma integrada.

\end{itemize}

%% ------------------------------------------------------------------------- %%
\section{Organização do Trabalho}
\label{sec:organizacao_trabalho}

No capítulo \ref{cap:conceitos}, são apresentados os conceitos mais elementares
de sismologia e de estatística relevantes para uma melhor compreensão do tema.
Em seguida, no capítulo \ref{cap:regiao_de_estudo} é apresentado com maior detalhe
a região de estudo sob o ponto de vista geológico e tectônico. No capítulo \ref{cap:teoria}
é apresentada e formalizada a teoria e os fundamentos dos métodos discutidos. O capítulo \ref{cap:processamento}
discorre sobre as etapas de processamento propriamente ditas, enquanto o capítulo \ref{cap:resultados}
foca apenas nos resultados de cada método e em resultados anteriores. As discussões relevantes a partir da análise
dos resultados é apresentada no capítulo \ref{cap:conclusoes} finalmente.
