%% ------------------------------------------------------------------------- %%
\chapter{Contexto Teórico}
\label{cap:teoria}


%% ------------------------------------------------------------------------- %%
\section{Apresentação}
%\index{área do trabalho!fundamentos}
\label{sec:c04_apresentacao}


Neste capítulo apresenta-se a formalização das metodologias aplicadas na fase 
processamento \ref{cap:processamento}. Essencialmente das \gls{smoothing} empregadas.

%% ------------------------------------------------------------------------- %%
\section{\Gls{smoothing}}
\index{suavização!fundamentos, metodologia}
\label{sec:04_smoothing_general}

As \gls{smoothing} em geral, permitem extrair feições importantes do conjunto de dados.

Quando aplicadas à caracterização das \glspl{seismic_source} em \gls{psha},
permitem gerar um conjunto de \glspl{point_source} 
caracterizadas pela determinação suavizada das \glspl{seismic_rate} 
nos pontos de uma malha regular. 

Na prática, a essencia da técnica é aplicar uma \gls{kernel_function} como na equação \ref{eq:kernel}

\begin{equation}
K(x)
\label{eq:kernel}
\end{equation}


Texto \ref{eq:kernel} texto texto texto texto texto texto texto texto texto
texto texto texto texto texto texto texto texto texto texto texto texto texto texto texto texto
texto texto texto texto texto texto texto texto texto texto texto texto texto

bla bla \eqref{eq:emc} \ldots


%% ------------------------------------------------------------------------- %%
\subsection{Frankel, 1995}\index{ácido!nucléico}\index{nucleotídeos}
\label{sec:acidos_nucleicos}

Texto texto texto texto texto texto texto texto texto texto texto texto texto
texto texto texto texto texto texto texto texto texto texto texto texto texto
texto texto texto texto texto texto texto texto texto texto texto texto texto
texto texto texto texto texto texto texto texto texto texto texto texto texto



%% ------------------------------------------------------------------------- %%
\subsection{Woo, 1996}\index{ácido!nucléico}\index{nucleotídeos}
\label{sec:acidos_nucleicos}

Texto texto texto texto texto texto texto texto texto texto texto texto texto
texto texto texto texto texto texto texto texto texto texto texto texto texto
texto texto texto texto texto texto texto texto texto texto texto texto texto
texto texto texto texto texto texto texto texto texto texto texto texto texto




\begin{equation}
	\ensuremath{
		\gls{sym:Rrm} = \sum_{i=1}^{N} \frac{ K(m, \gls{sym:r} - \gls{sym:ri})}
											{T({\gls{sym:ri})}}
	}
	\label{eq:Rrm}
\end{equation}
onde \ldots.


texto texto texto texto texto texto texto texto texto texto texto texto texto
texto texto texto texto texto texto texto texto texto texto texto texto texto

\begin{equation}
	\ensuremath{
		K(\gls{sym:r}, m) =  \frac{  \gls{sym:aW}  -1}{\pi\gls{sym:hm}^2}
							\left( 1 + \frac{\gls{sym:r}^2}{\gls{sym:hm}^2} \right)^{-\gls{sym:aW}}
	}
	\label{eq:/krm1}
\end{equation}
onde \ldots.


texto texto texto texto texto texto texto texto texto texto texto texto texto
texto texto texto texto texto texto texto texto texto texto texto texto texto



\begin{equation}
	\ensuremath{
		K(\gls{sym:r}, m) = 
		\begin{cases}
			\frac{\gls{sym:DW}}{2\pi \gls{sym:hm}^2} 
			\left( \frac{\gls{sym:hm}}{\gls{sym:r}} \right)^{2 - \gls{sym:DW}} 
			  & r \leq \gls{sym:hm} \\
			0 & r > \gls{sym:hm}
		\end{cases}
	}
	\label{eq:krm2}
\end{equation}
onde \ldots.


texto texto texto texto texto texto texto texto texto texto texto texto texto
texto texto texto texto texto texto texto texto texto texto texto texto texto


\begin{equation}
	\ensuremath{
		\gls{sym:hm} = ce^{dm}
	}
	\label{eq:krm2}
\end{equation}
onde $c$ e $d$ são determinados por regressão entre a 
distância média $h$ de cada tremor ao vizinho mais próximo em cada faixa de magnitude $m$.


texto texto texto texto texto texto texto texto texto texto texto texto texto
texto texto texto texto texto texto texto texto texto texto texto texto texto




%% ------------------------------------------------------------------------- %%
\subsection{Helmstetter, 2012}
\index{hemlstetter, 2012}
\label{sec:helmstetter}

Texto texto texto texto texto texto texto texto texto texto texto texto texto
texto texto texto texto texto texto texto texto texto texto texto texto texto

\begin{equation}
	\ensuremath{\gls{sym:R} = \sum_{i=1}^{N}{ \frac{1}{h_i {d_i}^2} \gls{sym:K1}\gls{sym:K2} }}
	\label{eq:helms01}
\end{equation}
onde \gls{sym:R} é \glsdesc{sym:R}, 
	  $K_1$ é o \glsdesc{sym:K1}, 
	  $K_2$ é o \glsdesc{sym:K2}.


texto texto texto texto texto texto texto texto texto texto texto texto texto
texto texto texto texto texto texto texto texto texto texto texto texto texto

\begin{equation}
\ensuremath{\gls{sym:R} = \gls{sym:Rmin} + \sum_{t_i < t}{ \frac{2}{h_i {d_i}^2} \gls{sym:K1}\gls{sym:K2} }}
	\label{eq:helms02}
\end{equation}
onde \gls{sym:Rmin} é a \glsdesc{sym:Rmin}.

texto texto texto texto texto texto texto texto texto texto texto texto texto
texto texto texto texto texto texto texto texto texto texto texto texto texto

\begin{equation}
	\ensuremath{ \gls{sym:wi} = 10^{ \gls{sym:b} \left( \gls{sym:Mc_rt} - \gls{sym:Md} \right) } }
	\label{eq:helms_wi}
\end{equation}
onde  \gls{sym:wi} é o \glsdesc{sym:wi}, 
	  \gls{sym:b} é o \glsdesc{sym:b}, 
	  \gls{sym:Mc_rt} é a \glsdesc{sym:Mc_rt}, 
	  \gls{sym:Md} é a \glsdesc{sym:Md}.

texto texto texto texto texto texto texto texto texto texto texto texto texto
texto texto texto texto texto texto texto texto texto texto texto texto texto

\begin{equation}
	\ensuremath{
		\underset{d_i \ge \gls{sym:dk} \\\ h_i \ge \gls{sym:hk}}{\argmin} 
		\left[ s\left(h_i,d_i\arrowvert\gls{sym:k_cnn},\gls{sym:a_cnn}\right) 
		:= h_i + \gls{sym:a_cnn}d_i \right]  
	}
	\label{eq:helms_cnn}
\end{equation}
onde \gls{sym:k_cnn} é \glsdesc{sym:k_cnn},
	 \gls{sym:a_cnn} é \glsdesc{sym:a_cnn},
	 \gls{sym:dk} é \glsdesc{sym:dk} e 
	 \gls{sym:hk} é \glsdesc{sym:hk}.


texto texto texto texto texto texto texto texto texto texto texto texto texto
texto texto texto texto texto texto texto texto texto texto texto texto texto


\begin{equation}
	\ensuremath{
		\gls{sym:L} = \sum_{i_x=1}^{N_x}\sum_{i_y=1}^{N_y}\log p\left[  \gls{sym:Np}, \gls{sym:nxy}  \right]
	}
	\label{eq:loglik}
\end{equation}
onde \ldots.


texto texto texto texto texto texto texto texto texto texto texto texto texto
texto texto texto texto texto texto texto texto texto texto texto texto texto


\begin{equation}
	\ensuremath{
		\gls{sym:pNn} = \frac{{N_p}^n e^{-N_p}}
							 {n!}
	}
	\label{eq:loglik}
\end{equation}
onde \ldots.


texto texto texto texto texto texto texto texto texto texto texto texto texto
texto texto texto texto texto texto texto texto texto texto texto texto texto


\begin{equation}
	\ensuremath{
		\gls{sym:G} = e^{ \frac{\gls{sym:L} - \gls{sym:Lu}}{\gls{sym:Nt}}   }
	}
	\label{eq:gain}
\end{equation}
onde \ldots.


texto texto texto texto texto texto texto texto texto texto texto texto texto
texto texto texto texto texto texto texto texto texto texto texto texto texto


\begin{equation}
	\ensuremath{
		\gls{sym:Lu} = -N_t + 
		\sum_{i_x = 1}^{N_x}\sum_{i_y=1}^{N_y}
		\gls{sym:nxy}\log N_u - \log \left[ \gls{sym:nxy}! \right]
	}
	\label{eq:lu}
\end{equation}
onde \ldots.


texto texto texto texto texto texto texto texto texto texto texto texto texto
texto texto texto texto texto texto texto texto texto texto texto texto texto


\begin{equation}
	\ensuremath{
	\begin{align}
		L - L_u & = \sum_{i_x = 1}^{N_x}\sum_{i_y=1}^{N_y}
				  \gls{sym:nxy}\log \left[ \frac{\gls{sym:Np}}{\gls{sym:Nu}} \right] \\
				& = \sum_{i = 1}^{N_t}\log \left[\frac{\gls{sym:Npi}}{\gls{sym:Nu}} \right]
	\end{align}}
	\label{eq:llu}
\end{equation}
onde \ldots.


texto texto texto texto texto texto texto texto texto texto texto texto texto
texto texto texto texto texto texto texto texto texto texto texto texto texto


\begin{equation}
	\ensuremath{
	\begin{align}
		G & = e^{\sum_{i = 1}^{N_t}
					\frac{\log \left[  \gls{sym:Npi} / \gls{sym:Nu}  \right]}
						 {\gls{sym:Nt}}
			  } \\
		  & = {\langle  \gls{sym:Npi} / \gls{sym:Nu}  \rangle}_{geom}
	\end{align}}
	\label{eq:G}
\end{equation}
onde \ldots.


texto texto texto texto texto texto texto texto texto texto texto texto texto
texto texto texto texto texto texto texto texto texto texto texto texto texto

\begin{equation}
	\ensuremath{
		\gls{sym:I} = \frac{1}{\gls{sym:Nt}} 
					  \sum_{i = 1}^{N_t}\log\left[ \frac{\gls{sym:NAi}}
					  								  {\gls{sym:NBi}}  \right]
	}
	\label{eq:gain}
\end{equation}
onde \ldots.


texto texto texto texto texto texto texto texto texto texto texto texto texto
texto texto texto texto texto texto texto texto texto texto texto texto texto





\begin{equation}
	\ensuremath{
		\sigma^2(x_i) = 	\frac{1}{N_t - 1}
					{\left(
					\sum_{i-1}^{N_t}
						{x_i}^2 
					\right)}
					- 
					\frac{1}{{N_t}^2 - N_t}
					{\left(
						\sum_{i=1}^{N_t}{x_i}
					\right)}^2
	}
	\label{eq:var}
\end{equation}
onde \ldots.


texto texto texto texto texto texto texto texto texto texto texto texto texto
texto texto texto texto texto texto texto texto texto texto texto texto texto




\begin{equation}
	\ensuremath{
		\gls{sym:T} = \frac{I\sqrt{N_t}}{\sigma}
	}
	\label{eq:T}
\end{equation}
onde \ldots.


texto texto texto texto texto texto texto texto texto texto texto texto texto
texto texto texto texto texto texto texto texto texto texto texto texto texto


