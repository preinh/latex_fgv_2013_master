%-----------------------------------------
% symbols
%-----------------------------------------

\newglossaryentry{sym:t}
{
	name={\ensuremath{t}},
	description={tempo},
	symbol={\ensuremath{t}},
	type=symbols
}

\newglossaryentry{sym:P}
{
	name={\ensuremath{P}},
	description={probabilidade},
	symbol={\ensuremath{P}},
	type=symbols
}

\newglossaryentry{sym:E}
{
	name={\ensuremath{E}},
	description={valor esperado},
	symbol={\ensuremath{E}},
	type=symbols
}

\newglossaryentry{sym:Var}
{
	name={\ensuremath{Var}},
	description={variança},
	symbol={\ensuremath{Var}},
	type=symbols
}

\newglossaryentry{sym:epsilon}
{
	name={\ensuremath{\epsilon}},
	description={erro},
	symbol={\ensuremath{\epsilon}},
	type=symbols
}

\newglossaryentry{sym:sigma}
{
	name={\ensuremath{\sigma}},
	description={desvio padrão},
	symbol={\ensuremath{\sigma}},
	type=symbols
}


\newglossaryentry{sym:r}
{
	name={\ensuremath{\boldsymbol{r}}},
	description={lugar no espaço},
	symbol={\ensuremath{\boldsymbol{r}}},
	type=symbols
}


\newglossaryentry{sym:m}
{
	name={\ensuremath{m}},
	description={magnitude},
	symbol={\ensuremath{m}},
	type=symbols
}


\newglossaryentry{sym:lambda}
{
	name={\ensuremath{\lambda}},
	description={função regressora para a taxa de sismicidade},
	symbol={\ensuremath{\lambda}},
	type=symbols
}

\newglossaryentry{sym:M_0}
{
	name={\ensuremath{M_0}},
	description={momento sísmico},
	symbol={\ensuremath{M_0}},
	type=symbols
}


\newglossaryentry{sym:mu}
{
	name={\ensuremath{\mu_{rig}}},
	description={coeficiente de rigidez da rocha},
	symbol={\ensuremath{\mu_{rig}}},
	type=symbols
}


\newglossaryentry{sym:A}
{
	name={\ensuremath{A}},
	description={área afetada},
	symbol={\ensuremath{A}},
	type=symbols
}


\newglossaryentry{sym:D}
{
	name={\ensuremath{\tilde{D}}},
	description={deslocamento médio},
	symbol={\ensuremath{\tilde{D}}},
	type=symbols
}


\newglossaryentry{sym:MW}
{
	name={\ensuremath{M_W}},
	description={magnitude de momento sísmico},
	symbol={\ensuremath{M_W}},
	type=symbols
}

\newglossaryentry{sym:A_richter}
{
	name={\ensuremath{\hat{A}}},
	description={amplitude no sismômetro Wood-Anderson},
	symbol={\ensuremath{\hat{A}}},
	type=symbols
}

\newglossaryentry{sym:d_richter}
{
	name={\ensuremath{\hat{d}}},
	description={distância entre o tremor e o sensor que era de aproximadamente 100km},
	symbol={\ensuremath{\hat{d}}},
	type=symbols
}


\newglossaryentry{sym:b}
{
	name={\ensuremath{b}},
	description={valor-b (corresponde à proporção de sismos pequenos e grandes, geralmente em torno de 1)},
	symbol={\ensuremath{b}},
	type=symbols
}


\newglossaryentry{sym:a}
{
	name={\ensuremath{a}},
	description={valor-a (corresponde à um índice de produtividade)},
	symbol={\ensuremath{a}},
	type=symbols
}


\newglossaryentry{sym:N_m}
{
	name={\ensuremath{N(m,m+\mathrm{d}m)}},
	description={número de eventos com magnitude entre $m$ e $m + \mathrm{d}m$ },
	symbol={\ensuremath{N(m)}},
	type=symbols
}


\newglossaryentry{sym:m_min}
{
	name={\ensuremath{m_{min}}},
	description={limite inferior da distribuição de magnitudes},
	symbol={\ensuremath{m_{min}}},
	type=symbols
}

\newglossaryentry{sym:m_max}
{
	name={\ensuremath{m_{max}}},
	description={limite superior da distribuição de magnitudes},
	symbol={\ensuremath{m_{max}}},
	type=symbols
}

\newglossaryentry{sym:m_c}
{
	name={\ensuremath{m_c}},
	description={magnitude de completude},
	symbol={\ensuremath{m_c}},
	type=symbols
}



\newglossaryentry{sym:m_corner}
{
	name={\ensuremath{m_{corner}}},
	description={valor de magnitude responsável por controlar o decaimento da Kagan-MFD },
	symbol={\ensuremath{m_{corner}}},
	type=symbols
}

\newglossaryentry{sym:M}
{
	name={\ensuremath{M}},
	description={variável aleatória representeando as magnitudes},
	symbol={\ensuremath{M}},
	type=symbols
}

\newglossaryentry{sym:beta}
{
	name={\ensuremath{\beta}},
	description={\beta = \gls{sym:b}\ln{10}},
	symbol={\ensuremath{\beta}},
	type=symbols
}

\newglossaryentry{sym:beta_p}
{
	name={\ensuremath{\beta_p}},
	description={$\beta_p = \frac{2}{3}\gls{sym:b}$, é o beta da distribuição de Pareto},
	symbol={\ensuremath{\beta_p}},
	type=symbols
}

\newglossaryentry{sym:alpha}
{
	name={\ensuremath{\alpha}},
	description={número total de sismos},
	symbol={\ensuremath{\alpha}},
	type=symbols
}


\newglossaryentry{sym:ri}
{
	name={\ensuremath{\boldsymbol{r}_i}},
	description={localização espacial do tremor $i$},
	symbol={\ensuremath{\boldsymbol{r}_i}},
	type=symbols
}


\newglossaryentry{sym:ti}
{
	name={\ensuremath{t_i}},
	description={localização temporal do tremor $i$},
	symbol={\ensuremath{t_i}},
	type=symbols
}


\newglossaryentry{sym:hi}
{
	name={\ensuremath{h_i}},
	description={largura de banda temporal para o tremor $i$},
	symbol={\ensuremath{h_i}},
	type=symbols
}


\newglossaryentry{sym:di}
{
	name={\ensuremath{d_i}},
	description={largura de banda espacial para o tremor $i$},
	symbol={\ensuremath{d_i}},
	type=symbols
}


\newglossaryentry{sym:wi}
{
	name={\ensuremath{ w }},
	description={peso},
	symbol={\ensuremath{ w }},
	type=symbols
}


\newglossaryentry{sym:Mc_rt}
{
	name={\ensuremath{ M_c\left( \gls{sym:r}, \gls{sym:t} \right)  }},
	description={magnitude de completude na localização \gls{sym:r} e no instante \gls{sym:t}},
	symbol={\ensuremath{ M_c\left( \gls{sym:r}, \gls{sym:t} \right) }},
	type=symbols
}


\newglossaryentry{sym:Mc}
{
	name={\ensuremath{M_c}},
	description={magnitude de completude},
	symbol={\ensuremath{M_c}},
	type=symbols
}

\newglossaryentry{sym:Md}
{
	name={\ensuremath{M_d}},
	description={valor mínimo de magnitude no catálogo},
	symbol={\ensuremath{M_d}},
	type=symbols
}


\newglossaryentry{sym:Rmin}
{
	name={\ensuremath{R_{min}}},
	description={mínima taxa de sismicidade},
	symbol={\ensuremath{R_{min}}},
	type=symbols
}


\newglossaryentry{sym:R}
{
	name={\ensuremath{R(\gls{sym:r},\gls{sym:t})}},
	description={taxa de sismicidade na localização \gls{sym:r} e no instante \gls{sym:t}},
	symbol={\ensuremath{R(\gls{sym:r},\gls{sym:t})}},
	type=symbols
}


\newglossaryentry{sym:Rrm}
{
	name={\ensuremath{R(\gls{sym:r},\gls{sym:m})}},
	description={taxa de sismicidade na localização \gls{sym:r} e no instante \gls{sym:t}},
	symbol={\ensuremath{R(\gls{sym:r},\gls{sym:t})}},
	type=symbols
}


\newglossaryentry{sym:Kt}
{
	name={\ensuremath{K_t \left( \frac{ t - \gls{sym:ti} }{ \gls{sym:hi} } \right) }},
	description={função de núcleo semi-gaussiana na dimensão do tempo, onde
					\gls{sym:ti} é a \glsdesc{sym:ti} e
					\gls{sym:hi} é a \glsdesc{sym:hi}
				},
	symbol={\ensuremath{K_t \left( \frac{ t - \gls{sym:ti} }{ \gls{sym:hi} } \right)}},
	type=symbols
}

\newglossaryentry{sym:Kr}
{
	name={\ensuremath{K_r \left( \frac{ \| \gls{sym:r} - \gls{sym:ri} \| }{d_i} \right) }},
	description={função de núcleo na dimensão do espaço, onde
					\gls{sym:ri} é a \glsdesc{sym:ri} e
					\gls{sym:di} é a \glsdesc{sym:di}
	},
	symbol={\ensuremath{K_r \left( \frac{ \| \gls{sym:r} - \gls{sym:ri} \| }{d_i} \right)}},
	type=symbols
}


\newglossaryentry{sym:Krm}
{
	name={\ensuremath{K(\gls{sym:r},\gls{sym:m})}},
	description={função de núcleo em uma certa localização \gls{sym:r} para sismos de magnitude \gls{sym:m}},
	symbol={\ensuremath{K_1 \left( \frac{ t - \gls{sym:ti} }{ \gls{sym:hi} } \right)}},
	type=symbols
}

\newglossaryentry{sym:a_cnn}
{
	name={\ensuremath{a_{cnn}}},
	description={acoplamento espaço-temporal},
	symbol={\ensuremath{a_{cnn}}},
	type=symbols
}

\newglossaryentry{sym:k_cnn}
{
	name={\ensuremath{k_{cnn}}},
	description={$k^{\'esimo}$ vizinho mais próximo},
	symbol={\ensuremath{k_{cnn}}},
	type=symbols
}


\newglossaryentry{sym:dk}
{
	name={\ensuremath{d_k}},
	description={$\max{\left\{ d_j \right\}}, j=1,\ldots,k_{cnn}$},
	symbol={\ensuremath{d_k}},
	type=symbols
}

\newglossaryentry{sym:hk}
{
	name={\ensuremath{h_k}},
	description={$\max{\left\{ h_j \right\} }, j=1,\ldots,k_{cnn}$},
	symbol={\ensuremath{h_k}},
	type=symbols
}

\newglossaryentry{sym:ixiy}
{
	name={\ensuremath{\left(i_x, i_y\right)}},
	description={cada célula do grid},
	symbol={\ensuremath{\left(i_x, i_y\right)}},
	type=symbols
}


\newglossaryentry{sym:N}
{
	name={\ensuremath{N}},
	description={número de tremores no catálogo/catálogo-teste},
	symbol={\ensuremath{N}},
	type=symbols
}



\newglossaryentry{sym:Np}
{
	name={\ensuremath{N_p\left(i_x, i_y\right)}},
	description={taxa de sismicidade prevista pelo modelo para a célula \gls{sym:ixiy}},
	symbol={\ensuremath{N_p\left(i_x, i_y\right)}},
	type=symbols
}


\newglossaryentry{sym:Nu}
{
	name={\ensuremath{N_u}},
	description={\gls{sym:Nt}/\gls{sym:Nc}},
	symbol={\ensuremath{N_u}},
	type=symbols
}


\newglossaryentry{sym:Nc}
{
	name={\ensuremath{N_c}},
	description={número de células da malha de análise},
	symbol={\ensuremath{N_c}},
	type=symbols
}


\newglossaryentry{sym:Npi}
{
	name={\ensuremath{N_p(i)}},
	description={a taxa de sismicidade predita pelo modelo no segmento espacial (\emph{spatial bin}) onde o tremor $i$
	ocorreu}, symbol={\ensuremath{N_p(i)}},
	type=symbols
}


\newglossaryentry{sym:NAi}
{
	name={\ensuremath{N_A(i)}},
	description={taxa de sismicidade em $i$ predita pelo modelo $A$}, 
	symbol={\ensuremath{N_A(i)}},
	type=symbols
}


\newglossaryentry{sym:NBi}
{
	name={\ensuremath{N_B(i)}},
	description={taxa de sismicidade em $i$ predita pelo modelo $B$},
	symbol={\ensuremath{N_B(i)}},
	type=symbols
}


\newglossaryentry{sym:Ts}
{
	name={\ensuremath{T_s}},
	description={valor-$T$ com distribuição de Student},
	symbol={\ensuremath{T_s}},
	type=symbols
}

\newglossaryentry{sym:nxy}
{
	name={\ensuremath{n\left(i_x, i_y\right)}},
	description={número de eventos observados na célula \gls{sym:ixiy}},
	symbol={\ensuremath{n\left(i_x, i_y\right)}},
	type=symbols
}


\newglossaryentry{sym:Nt}
{
	name={\ensuremath{N_t}},
	description={número de eventos no catálogo-alvo},
	symbol={\ensuremath{N_t}},
	type=symbols
}



\newglossaryentry{sym:L}
{
	name={\ensuremath{L}},
	description={log da máxima verossimilhança},
	symbol={\ensuremath{L}},
	type=symbols
}


\newglossaryentry{sym:Lu}
{
	name={\ensuremath{L_u}},
	description={log da máxima verossimilhança de um modelo uniforme},
	symbol={\ensuremath{L}},
	type=symbols
}


\newglossaryentry{sym:pNn}
{
	name={\ensuremath{p(N_p, n)}},
	description={probabilidade de se observar $n$ eventos com probabilidade \gls{sym:Np}},
	symbol={\ensuremath{p(N_p, n)}},
	type=symbols
}


\newglossaryentry{sym:G}
{
	name={\ensuremath{G}},
	description={ganho de probabilidade por cada tremor no catálogo-alvo
				 sobre um modelo espacialmente uniforme de Poisson.}, 
	symbol={\ensuremath{G}}, 
	type=symbols
}


\newglossaryentry{sym:I}
{
	name={\ensuremath{ I_{inf}(A,B)}},
	description={ganho de informação do modelo $A$ sobre o modelo $B$}, 
	symbol={\ensuremath{I_{inf}(A,B)}}, 
	type=symbols
}


\newglossaryentry{sym:aW}
{
	name={\ensuremath{a_W}},
	description={parâmetro de decaimento tipicamente entre 1.5 e 2 
				 que gera um decaimento de $3^{a}$ a $4^{a}$ ordem 
				 na densidade de probabilidade com a distância epicentral}, 
	symbol={\ensuremath{a_W}}, 
	type=symbols
}



\newglossaryentry{sym:DW}
{
	name={\ensuremath{D_W}},
	description={dimensão fractal dos epicentros $D_W = 2-\gls{sym:aW}$}, 
	symbol={\ensuremath{D_W}}, 
	type=symbols
}



\newglossaryentry{sym:hm}
{
	name={\ensuremath{h(m)}},
	description={largura de banda fixa para a magnitude $m$}, 
	symbol={\ensuremath{h(m)}}, 
	type=symbols
}



\newglossaryentry{sym:a0}
{
	name={\ensuremath{a_0}},
	description={parâmetro de \gls{sym:hm}}, 
	symbol={\ensuremath{a_0}}, 
	type=symbols
}



\newglossaryentry{sym:a1}
{
	name={\ensuremath{a_1}},
	description={parâmetro de \gls{sym:hm}}, 
	symbol={\ensuremath{a_1}}, 
	type=symbols
}




\newglossaryentry{sym:dF}
{
	name={\ensuremath{d_F}},
	description={distância de correlação}, 
	symbol={\ensuremath{d_F}}, 
	type=symbols
}


\newglossaryentry{sym:dij}
{
	name={\ensuremath{d_{ij}}},
	description={distância entre a célula $i$ e a célula $j$ na malha}, 
	symbol={\ensuremath{d_{ij}}}, 
	type=symbols
}


\newglossaryentry{sym:I0}
{
	name={\ensuremath{I_0}},
	description={intensidade máxima}, 
	symbol={\ensuremath{I_0}}, 
	type=symbols
}


\newglossaryentry{sym:Af}
{
	name={\ensuremath{A_{f}}},
	description={área afetada medida em km$^2$}, 
	symbol={\ensuremath{A_{f}}}, 
	type=symbols
}



